\section{Auswertung}

\begin{equation}
  B(z) = az^2 + bz + c
\end{equation}
\begin{align*}
  a &= \SI{-1.31 \pm 0.08}{\frac{mT}{mm^2}} \\
  b &= \SI{2.99 \pm 0.90}{\frac{mT}{mm}} \\
  c &= \SI{405 \pm 10}{mT}
\end{align*}
Maximum bei:
\begin{equation*}
  z_\text{max} = -\frac{b}{2a} \implies B_\text{max} = -\frac{b^2}{4a} + c = \SI{408 \pm 10}{mT}
\end{equation*}

\begin{table}
  \centering
  \caption{Magnetfeldstärke entlang des Strahlengangs.}
  \label{tab:B}
  \begin{tabular}{S[table-format=3] S[table-format=+2]}
    \toprule
    {$B\:/\:$mT} & {$z\:/\:$mm} \\
    \midrule
     36 & -16 \\
    162 & -11 \\
    355 & -06 \\
    387 & -04 \\
    404 & -02 \\
    410 & -01 \\
    411 & 0 \\
    410 & 2 \\
    402 & 4 \\
    382 & 6 \\
    345 & 8 \\
    317 & 9 \\
    124 & 14 \\
     29 & 19 \\
    \bottomrule
  \end{tabular}
\end{table}

\begin{figure}
  \centering
  \includegraphics{plots/plot_B.pdf}
  \caption{Messwerte und Fit der Magnetfeldstärke entlang des Strahlengangs.}
  \label{fig:B}
\end{figure}

\begin{figure}
  \centering
  \includegraphics{plots/plot_GaAs1.pdf}
  \caption{Die normierten Messwerte der Faraday-Rotation.}
  \label{fig:GaAs1}
\end{figure}

\begin{table}
  \centering
  \caption{Messwerte zum hochreinen GaAs.}
  \label{tab:rein}
  \begin{tabular}{S[table-format=1.3] S[table-format=2.4] S[table-format=4.2]}
    \toprule
    {$\lambda\:/\:\si{\micro\meter}$} & {$\theta\:/\:\si{\degree}$} & {$\frac{\theta}{d}\:/\:\si{\degree m^{-1}}$} \\
    \midrule
    1,06  & 7.4500  & 1457.93 \\
    1,29  & 6.7167  & 1314,42 \\
    1,45  & 4.0333  & 789,30 \\
    1,72  & 5.4667  & 1069,80 \\
    1,96  & 1.4583  & 285,38 \\
    2,156 & 7.5500  & 1477,50 \\
    2,34  & 13.9500 & 2729,94 \\
    2,51  & 1.6750  & 327,79 \\
    \bottomrule
  \end{tabular}
\end{table}

\begin{table}
  \centering
  \caption{Messwerte zum n-dotierten GaAs mit $N = \SI{1.2e18}{cm^{-3}}$.}
  \label{tab:rein}
  \begin{tabular}{S[table-format=1.3] S[table-format=2.4] S[table-format=5.2]}
    \toprule
    {$\lambda\:/\:\si{\micro\meter}$} & {$\theta\:/\:\si{\degree}$} & {$\frac{\theta}{d}\:/\:\si{\degree m^{-1}}$} \\
    \midrule
    1,06  & 4.4333  & 3259,78 \\
    1,29  & 4.7417  & 3486,54 \\
    1,45  & 2.4250  & 1783,09 \\
    1,72  & 1.0250  & 753,68 \\
    1,96  & 3.5000  & 2573,53 \\
    2,156 & 5.6083  & 4123,75 \\
    2,34  & 15.0533 & 11068,60 \\
    2,51  & 1.2833  & 943,60 \\
    \bottomrule
  \end{tabular}
\end{table}

\begin{table}
  \centering
  \caption{Messwerte zum n-dotierten GaAs mit $N = \SI{2.8e18}{cm^{-3}}$.}
  \label{tab:rein}
  \begin{tabular}{S[table-format=1.3] S[table-format=2.4] S[table-format=4.2]}
    \toprule
    {$\lambda\:/\:\si{\micro\meter}$} & {$\theta\:/\:\si{\degree}$} & {$\frac{\theta}{d}\:/\:\si{\degree m^{-1}}$} \\
    \midrule
    1,06  & 5.7583 & 4443,13 \\
    1,29  & 2.0500 & 1581,79 \\
    1,45  & 4.1333 & 3189,27 \\
    1,72  & 2.4833 & 1916,13 \\
    1,96  & 4.1500 & 3202,16 \\
    2,156 & 8.4167 & 6494,37 \\
    2,34  & 4.7000 & 3626,54 \\
    2,51  & 7.9583 & 6140,66 \\
    \bottomrule
  \end{tabular}
\end{table}


\begin{figure}
  \centering
  \includegraphics{plots/plot_GaAs2.pdf}
  \caption{Die Differenzen der Messwerte der Faraday-Rotation.}
  \label{fig:GaAs2}
\end{figure}

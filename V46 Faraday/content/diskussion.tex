\section{Diskussion}
Zunächst ist anzumerken, dass die Justierung des Versuchsaufbaus nicht gut funktioniert hat. Auffällig war, dass die Lichtintensität
an dem einen Photodetektor nie ganz verschwunden ist, wohingegen die Intensität am anderen Detektor verschwand, wenn der Winkel des
ersten Prismas entsprechend eingestellt war. Es konnte  dadurch keine Nulllinie am Oszilloskop erzeugt
werden, sondern es blieb immer eine Spannung im Bereich von ein paar Volt übrig. Das bedeutet, dass während der Durchführung immer nur ein
Minimum abgeschätzt werden konnte, wodurch der Ablesefehler sehr groß ist. Dies zeigt sich in der nahezu wahllosen Verteilung einzelner
Messwerte, wie in Abbildungen \ref{fig:GaAs1} und \ref{fig:GaAs2} zu erkennen ist.

In Anbetracht dieser großen systematischen Fehler ist es verwunderlich, dass die Ergebnisse für die effektive Masse wenig vom
Literaturwert $m^*_\text{lit} = \num{0.067}m_\text{e}$ \cite{meff} abweichen. Die relativen Fehler betragen:
\begin{align*}
  N = \SI{1.2e18}{cm^{-3}} &\text{:}\quad \symup{\Delta}m^*= \SI{-20\pm40}{\percent} \\
  N = \SI{2.8e18}{cm^{-3}} &\text{:}\quad \symup{\Delta}m^*= \SI{20\pm40}{\percent}
\end{align*}

\section{Auswertung}
\label{sec:auswertung}
Die in der Auswertung angegebenen Größen sind stets auf die erste signifikante
Stelle der Unsicherheit gerundet. Setzt sich eine Größe über mehrere Schritte aus
anderen Größen zusammen, so wird erst am Ende gerundet, um Fehler zu minimieren.
Zur Auswertung wird die Programmiersprache \texttt{python (Version 3.6.5)} mit
den Bibliothekserweiterungen \texttt{numpy}~\cite{numpy},
\texttt{scipy}~\cite{scipy} und \texttt{matplotlib}~\cite{matplotlib} verwendet.\\
\newline
Da es sich bei den aufgenommenen Werten um Zählwerte handelt, sind diese mit einem
Poissonfehler behaftet. Dadurch wird bei ~$N$ erfassten Spannungsimpulsen eine Unsicherheit
von~$\sqrt{N}$ angenommen.

\subsection{Justierung und Überprüfung der Koinzidenzapparatur}
Die zwischen die Sekundärelektronenvervielfacher und die Koinzidenzapparatur
geschalteten Diskriminatoren liefern ab einer nicht näher bestimmten eingehenden
Spannungsschwelle~\SI{20}{\nano\second} breite Spannungsimpulse.\\
\newline
Zur Bestimmung der optimalen Verzögerungzeit, sodass die Zählrate maximiert wird,
wird die Zählrate in Abhängigkeit der Verzögerungszeit bestimmt. Dazu wird an der
einen Verzögerungsleitung eine feste Verzögerung von~\SI{20}{\nano\second} eingestellt
und die zweite von $0$ bis \SI{40}{\nano\second} variiert.
Die dabei aufgenommenen Messwerte, das heißt die in einer Zeitspanne von~\SI{15}{\second}
gemessene Anzahl an Spannungsimpulsen, sind in Tabelle~\ref{tab:verzoegerung} aufgeführt.
Abbildung~\ref{fig:verzoegerung} zeigt die berechnete Zählrate aufgetragen gegen
die Verzögerungszeit. Es zeigt sich, dass eine relative Verzögerung
von~$T_{\symup{VZ}}=\SI{4}{\nano\second}$ zu einer maximalen Zählrate führt.
Die Auflösungszeit~$\symup{\Delta}t_K$ der Koinzidenzapparatur ist gleichbedeutend
mit der in grün eingezeichneten Halbwertsbreite. Zur Bestimmung von dieser wird
ein linearer Fit für das Plateau und die beiden Flanken durchgeführt. Durch die halbe
Höhe des Plateau ist anschließend die Halbwertsbreite gegeben. Durch die Schnittpunkte
mit den beiden Flanken lässt sich nun die Verzögerungszeit zu
\begin{equation}
  \symup{\Delta}t_K = |-28| + |11| = \SI{39}{\nano\second}
\end{equation}
bestimmen.
Die Gesamtbreite der beiden Diskriminatoren beträgt \SI{40}{\nano\second}, dies
führt zu einer Abweichung von \SI{2.5}{\percent}.

\begin{table}[htb]
  \centering
  \caption{Messwerte zur Bestimmung der optimalen Verzögerungszeit.}
  \begin{tabular}{S[table-format = 2.1] S[table-format = 3(2)] |
                  S[table-format = 2.1] S[table-format = 3(2)] }
    \toprule
    {$t_{\symup{VZ}}$ in \si{\nano\second}} & {Impulse in \SI{15}{\second}} & {$t_{\symup{VZ}}$ in \si{\nano\second}} & {Impulse in \SI{15}{\second}} \\
    \midrule
    -20.0 & 586(24) &   2.0 & 646(25) \\
     -18.0 & 593(24) &  4.0 & 621(25) \\
     -16.0 & 579(24) &  6.0 & 616(25) \\
     -14.0 & 599(24) &  8.0 & 668(26) \\
     -12.0 & 605(24) &  10.0 & 627(25) \\
     -10.0 & 599(24) &  12.0 & 620(25) \\
     -8.0 & 562(24) &   14.0 & 639(25) \\
     -6.0 & 620(25) &   16.0 & 597(24) \\
     -4.0 & 577(24) &   18.0 & 582(24) \\
     -2.0 & 577(24) &   20.0 & 568(24) \\
      0.0 & 574(24) &   &  \\
    \bottomrule
  \end{tabular}
  \label{tab:verzoegerung}
\end{table}

\begin{figure}[htb]
  \centering
  \includegraphics[width=0.7\textwidth]{plots/Plateau.pdf}
  \caption{Messwerte zur Bestimmung der optimalen Verzögerungszeit.}
  \label{fig:verzoegerung}
\end{figure}

\subsection{Zeitkalibrierung des Vielkanalanalysators}
Mit Hilfe eines Doppelimpulsgenerators wird eine Kalibrierung der Zeitachse des
Vielkanalanalysators durchgeführt. Der Doppelimpulsgenerator
liefert mit einer hohen Rate je zwei Spannungsimpulse, deren zeitlicher Abstand
über einen Kodierschalter festgelegt werden kann. Die Zeitdifferenzen, sowie die
jeweils ansprechenden Kanäle sind in Tabelle~\ref{tab:kalibrierung} aufgeführt.
Es zeigt sich, dass für eine Zeitdifferenz häufig nicht nur ein Kanal, sondern
auch ein direkt benachbarter Kanal anspricht. Dies ist vermutlich auf
Störungen in der Übertragungsleitung des Vielkanalanalysators zurückzuführen.
Um dennoch einem Kanal eine bestimmte Zeitdifferenz zuordnen zu können, wird ein
gewichteter Mittelwert~$\bar{x}_i$ aller bei einer bestimmten Zeitdifferenz~$\symup{\Delta}t_i$
ansprechenden Kanäle bestimmt.

\begin{equation}
  \bar{x}_i=\frac{1}{N_i}\sum_{j=1}^{N_i}n_{ij}x_{ij}.
  \label{eq:kanal_mittelwert}
\end{equation}

Hierbei ist~$N_i$ die Gesamtzahl an registrierten Ereignissen für eine bestimmte
Zeitdifferenz. $n_{ij}$ ist die Anzahl der Ereignisse für einen bestimmten
Kanal~$x_{ij}$. Weiterhin ergibt sich für jeden Mittelwert eine
Standardabweichung~$\symup{\Delta}\bar{x}_i$:

\begin{equation}
  \symup{\Delta}\bar{x}_i=\sqrt{\frac{1}{N_i}\sum_{j=1}^{N_i}n_{ij}\left(x_{ij}-\bar{x}_i\right)^2}
  \label{eq:kanal_standardabweichung}
\end{equation}

Für eine eindeutige Zuordnung einer Zeitdifferenzen zu einem entsprechenden Kanal,
wird eine lineare Regression durch die gemessenen Stützstellen durchgeführt. Diese
liefert die Gerade

\begin{equation}
  \symup{\Delta}t=\SI{0.04541(21)}{\micro\second}\cdot k+\SI{0.06(3)}{\micro\second}
  \label{eq:kalibrierung}
\end{equation}

wobei~$\symup{k}$ für die Nummer des Kanals steht.\footnote{Die lineare
Regression wird mit Hilfe der Funktion~\texttt{curve\_fit} aus der
Bibliothek~\texttt{scipy.optimize} durchgeführt. Die Regression nur mit den
Mittelwerten durchgeführt. Die Unsicherheiten der Parameter ergeben sich dann ausschließlich
aus den Fehlern der Regressionrechnung.} Abbildung~\ref{fig:kalibrierung} zeigt
die aufgenommen Messwerte sowie die berechnete Regressionsgerade. Für die Kanäle
außerhalb der gemessenen Stützstellen ergibt sich die Zeitdifferenz durch Extrapolation
der Regressionsgeraden. Es wird angenommen, dass auch die Kanäle mit $\symup{k}>200$
dem linearen Zusammenhang folgen.

\begin{table}[htb]
  \centering
  \caption{Messwerte zur Zeikalibrierung des Vielkanalanalysators.}
  \begin{tabular}{S[table-format = 2.1] S[table-format = 3.1(1)]}
    \toprule
    {$\symup{\Delta}t$ in \si{\micro\second}} & {Zugeordneter Kanal} \\
    \midrule
    1.0 &  23.1(1) \\
    2.0 &  45.5(1) \\
    3.0 &  67.6(1) \\
    4.0 &  89.6(1) \\
    5.0 & 111.7(1) \\
    6.0 & 133.8(1) \\
    7.0 & 153.6(2) \\
    8.0 & 177.9(5) \\
    9.0 & 200.0(1) \\
    \bottomrule
  \end{tabular}
  \label{tab:kalibrierung}
\end{table}

\begin{figure}[htb]
  \centering
  \includegraphics[width=0.7\textwidth]{plots/Kanal.pdf}
  \caption{Messung zur Zeitkalibrierung des Vielkanalanalysators. Neben den
  Messwerten ist auch die berechnete Regressionsgerade aufgetragen. Bei der
  Berechnung der Regressionsgeraden werden die Fehler auf die Kanalnummern nicht
  berücksichtigt.}
  \label{fig:kalibrierung}
\end{figure}

\subsection{Abschätzung des Untergrundes}
Insgesamt wurden~$N_{\symup{S}}=\num{3500949(1871)}$ Startimpulse bei einer
Messzeit von~\SI{162391}{\second} registriert. Die daraus resultierende Rate
beträgt

\begin{equation}
  \langle N_{\symup{S}}\rangle=\SI{21.559(12)}{\per\second}.
  \label{eq:rate_start}
\end{equation}

Die Anzahl der Untergrundereignisse bestimmt sich aus der Wahrscheinlichkeit,
dass innerhalb der Suchzeit, die durch ein Myon gestartet wurde, ein weiteres
Myon vom Detektor registriert wird. Dabei folgt die Wahrscheinlichkeit einer
Poissonverteilung. Die Suchzeit wird am Gerät
auf~$T_{\symup{S}}=\SI{20}{\micro\second}$ eingestellt. Die Betrachtung der
Messwerte zeigt jedoch, dass ab einer Zeitdifferenz
von~$T_{\symup{S}}=\SI{19.47(9)}{\micro\second}$ (dies entspricht nach
Gleichung~\eqref{eq:kalibrierung} dem Kanal~\num{430}) keine Ereignisse mehr
detektiert werden. Somit ist davon auszugehen, dass es sich hierbei um die
tatsächliche Suchzeit handelt. Die Anzahl~$N_{\symup{F}}$ der registrierten
Fehlereignisse unterscheidet sich auf Grund der geringen Abweichung aber nicht und
ergibt sich zu

\begin{equation}
  N_{\symup{F}}(k=1)=N_{\symup{S}}\cdot\frac{\left(T_{\symup{S}}\langle N_{\symup{S}}\rangle\right)^k}{k!}\exp\left(T_{\symup{S}}\langle N_{\symup{S}}\rangle\right)=\num{1508(16)}.
  \label{eq:untergrund_gesamt}
\end{equation}

Unter der Annahme, dass sich der erwartete Untergrund statistisch auf alle
Kanäle gleichverteilt, lässt sich der Untergrund pro Kanal bestimmen. Dabei
werden bei der Berechnung die Kanäle~\num{431} bis~\num{511} nicht berücksichtigt,
da sie nicht signifikant zum exponentiellen Verlauf beitragen und sich vermutlich
auf statistische Schwankungen zurückführen lassen. Umgerechnet auf die verbleibenden~\num{430} Kanäle
ergibt sich der erwartete mittlere Untergrund pro Kanal zu

\begin{equation}
  U_0=\num{3.509(4)}
  \label{eq:untergrund_kanal}
\end{equation}

Die Unsicherheit von~$U_0$ wird im Folgenden aufgrund der geringen Größe
vernachlässigt.

\subsection{Bestimmung der mittleren Lebensdauer eines Myons}
Die vom Vielkanalanalysator erfassten Messwerte sind im Anhang aufgelistet.
Die Abbildung~\ref{fig:spektrum1} zeigt die Messwerte in einem Histogramm. Um nun
die mittlere Lebensdauer~$\tau$ eines Myons bestimmen zu können, wird eine Funktion
der Form

\begin{equation}
  N(t)=N_0\exp\left(-\frac{t(k)}{\tau}\right)+U_0
  \label{eq:fitfunktion}
\end{equation}

an die Messwerte angepasst. Hierbei ist~$k$ die Kanalnummer. Die
Anpassung im Bezug auf die Kanalnummern erweist sich als vorteilhaft, da bei
dieser keine Unsicherheiten in den~$x$-Werten auftreten. Für die Anpassung werden die
Kanäle~\num{0} bis~\num{2} ausgeschlossen, da sie stark vom exponentiellen Verlauf
abweichen und somit als Folge eines fehlerhaften Versuchsaufbaus interpretiert
werden können. Eine weitere Diskussion der Abweichungen findet sich in
Abschnitt~\ref{sec:diskussion}. Die Anpassung der
verbleibenden Messwerte an die Funktion in Gleichung~\ref{eq:fitfunktion}
liefert die Parameter

\begin{align}
  N_0&=\num{303(5)} \\
  U_0&=\num{2.2(10)}\\
  \label{eq:fitparameter}
\end{align}

Die mittlere Lebensdauer~$\tau$ eines Myons ergibt sich zu:

\begin{equation}
  \tau=\SI{2.252(62)}{\micro\second}.
  \label{eq:ergebnis}
\end{equation}

Die Abbildung~\ref{fig:spektrum3} zeigt die Ausgleichsfunktion zusammen mit den
Messwerten.

\begin{figure}[htb]
  \centering
  \includegraphics[width=0.7\textwidth]{plots/spektrum1.pdf}
  \caption{Häufigkeitsverteilung der gemessenen Zerfallszeiten, ausgedrückt
  durch die entsprechenden Kanäle des Vielkanalanalysators.}
  \label{fig:spektrum1}
\end{figure}

\begin{figure}[htb]
  \centering
  \includegraphics[width=0.7\textwidth]{plots/spektrum1_fit.pdf}
  \caption{Messwerte zur Berechnung der mittleren Lebensdauer von Myonen und
  Ausgleichsfunktion.}
  \label{fig:spektrum3}
\end{figure}

\section{Einleitung}
Ziel des Versuchs ist die Messung der charakteristischen Lebensdauer von Myonen, indem Individuallebensdauern
einzelner Myonen gemessen werden.

\section{Theorie}
\subsection{Elementarteilchen nach dem Standardmodell}
\label{sec:1}
Nach dem Standardmodell gibt es zwei verschiedene Elementarteilchen, die Quarks und die Leptonen. Beide lassen sich in
drei Generationen einteilen. Quarks unterliegen der starken Wechselwirkung, während Leptonen nur der schwachen Wechselwirkung
unterliegen. Leptonen sind Fermionen, also Spin-$\frac{1}{2}$-Teilchen, die der Fermi-Dirac-Statistik folgen.
Die drei Generationen der Leptonen unterscheiden sich durch ihre Masse und haben eine unterschiedliche Lebensdauer.
Elektronen gehören zur ersten Leptonen-Generation, während die hier betrachteten Myonen zur zweiten Generation gehören und in etwa
die $\num{206.77}$-fache Masse der Elektronen besitzen. Im Gegensatz zu Elektronen sind Myonen (und Tauonen) nicht stabil, sondern
haben eine endliche Lebensdauer.

Die hier betrachteten Myonen entstehen in der höheren Erdatmosphäre. Zunächst treffen hochenergetische Protonen aus der Höhenstrahlung
auf Atomkerne der Luftmoleküle und erzeugen Pionen. Diese haben nur eine sehr kurze Lebensdauer und zerfallen dann in Myonen:
\begin{align*}
  \uppi^+ &\to \upmu^+ + \upnu_{\upmu} \\
  \uppi^- &\to \upmu^- + \bar{\upnu}_{\upmu}\,.
\end{align*}
Da die Lebensdauer der Myonen deutlich größer ist und die Myonen sich nahezu mit Lichtgeschwindigkeit bewegen, gelangen diese
bis zur Erdoberfläche und können dort in einem Szintillator detektiert werden.

In der Szintillatormaterie wechselwirken die Myonen mit den Molekülen und geben dabei mehrere MeV ihrer kinetischen Energie in kleinen
Bruchteilen an die Moleküle ab. Diese werden dadurch angeregt und fallen nach einer Weile wieder in ihren Grundzustand zurück, wobei
sie ein Photon aussenden. Bei Eintreffen eines Myons entsteht also eine große Anzahl Photonen mit Energie im Bereich des sichtbaren
Lichts. Einige Myonen haben eine so geringe Energie, dass sie im Szintillator komplett abgebremst werden und dort nach einer Weile
zerfallen. Die Myonen zerfallen folgendermaßen in Elektronen und Neutrinos:
\begin{equation*}
  \upmu^- \to \symup{e}^- + \bar{\upnu}_\text{e} + \upnu_{\upmu} \,.
\end{equation*}
Das entsprechende Antiteilchen, das Antimyon, verhält sich genauso wie das Myon und zerfällt entsprechend in ein Positron und Neutrinos:
\begin{equation*}
  \upmu^+ \to \symup{e}^+ + \upnu_\text{e} + \bar{\upnu}_{\upmu} \,.
\end{equation*}
Die entstehenden Elektronen und Positronen erzeugen ebenfalls einen Lichtblitz im Szintillator, sodass der zeitliche Abstand zwischen
den beiden Lichtblitzen der individuellen Lebensdauer des detektierten Myons entspricht.

\subsection{Bestimmung der mittleren Lebensdauer von Elementarteilchen}
Da es sich beim Zerfall des Myons um einen statistischen Prozess handelt, besitzen einzelne Myonen unterschiedliche individuelle Lebensdauern,
sodass eine allgemeinere Definition der Lebensdauer nötig ist, die mittlere Lebensdauer. Da die Zerfälle der einzelnen Teilchen unabhängig voneinander sind,
ergibt sich bei einer Gesamtteilchenzahl $N$ für die Anzahl $\symup{d}N$ der Teilchen, die in der Zeit $\symup{d}t$ zerfallen
\begin{equation}
  \symup{d}N = -N\symup{d}W = -\lambda N\symup{d}t \,,
\end{equation}
wobei $\symup{d}W = \lambda\symup{d}t$ die Zerfallswahrscheinlichkeit eines einzelnen Teilchens im Zeitintervall $\symup{d}t$ ist und $\lambda$ die Zerfallskonstante ist.
Für eine sehr große Teilchenzahl $N$ kann dieser Zusammenhang näherungsweise integriert werden:
\begin{equation}
  \frac{N(t)}{N_0} = \text{e}^{-\lambda t} \,.
\end{equation}
Daraus ergibt sich die exponentielle Verteilungsfunktion der Lebensdauer $t$:
\begin{equation}
  \frac{\symup{d}N(t)}{N_0} = \lambda\text{e}^{-\lambda t}\symup{d}t\,.
  \label{eq:verteilung}
\end{equation}
Die mittlere Lebensdauer ergibt sich als Mittelwert aus allen möglichen Lebensdauern gewichtet mit der Verteilungsfunktion. Dies entspricht dem Erwartungswert
\begin{equation}
  \text{E}x := \int_{-\infty}^\infty xf(x)\symup{d}x
\end{equation}
der Verteilungsfunktion \eqref{eq:verteilung}.
Es ergibt sich:
\begin{equation}
  \tau = \int_{-\infty}^\infty \lambda t \text{e}^{-\lambda t} \symup{d}t = \frac{1}{\lambda} \,,
\end{equation}
die mittlere Lebensdauer $\tau$ entspricht also der inversen Zerfallskonstante $\lambda$.

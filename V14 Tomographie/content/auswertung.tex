\section{Auswertung}
Zu Beginn des Experiments wird der statistische Poissonfehler mit Hilfe von:
\begin{equation*}
  \frac{1}{\sqrt{N}} < 0.03
\end{equation*}
so abgeschätzt, dass er unterhalb von $<\SI{3}{\percent}$ liegt und somit für die weitere
Rechnung vernachlässigt werden kann. Dies ist der Fall, sobald $\text{N}>1112$
ist und wird für die gesamten Messungen eingehalten. Um die Aktivität der verwendetet Quelle zu bestimmen, wird zunächst eine Messung durchgeführt bei der sich kein Würfel im Strahlengang befindet. Dabei entsteht das in Abbildung \ref{fig:leer} dargestellte Spektrum. Dies führt bei einer Messung von $13708$ counts über $\SI{74.46}{\second}$ zu einer Rate von $I'_0 = 225.4\,\pm\,0.7\:\frac{\text{counts}}{\text{s}}$.

\begin{figure}
  \centering
  \includegraphics{plots/leer.pdf}
  \caption{Energieverteilung der Leermessung ohne Würfel im Strahlengang. Bei den schwarzen Fehlerbalken der Bins handelt es sich um statistische Fehler.}
  \label{fig:leer}
\end{figure}

Da sich die einzelnen Würfel von einem Aluminiumgehäuse zusammen gehalten werden, reicht dieser Wert nicht aus um eine Referenz für die Absorption zu bieten. Deshalb wird zusätzlich die Absorption durch den leeren Würfel bestimmt.
Die Abbildung~\ref{fig:alu} zeigt das gemessene Spektrum.

\begin{figure}
  \centering
  \includegraphics{plots/plot.pdf}
  \caption{Energieverteilung der Messung mit dem Aluminiumgehäuse im Strahlengang. Bei den schwarzen Fehlerbalken der Bins handelt es sich um statistische Fehler.}
  \label{fig:alu}
\end{figure}

Da es unterschiedliche Weglängen in dem Würfel gibt, bietet es sich an über alle möglichen verschiedenen Weglängen in dem Gehäuse zu messen, sodass man einen möglichst exakten Vergleichswert für jeden Strahlengang im Würfel hat. Dazu wurden die Projektionen 2, 9, 10 und 11 vermessen. Dadurch ergibt sich der Intensitätsvektor~\ref{eq:int} für die gemessenen Werte und die Vergleichswerte mit den entsprechend gleichen Strahlengängen in Reihenfolge der in Abbildung~\ref{fig:wuerfel} dargestellten Projektionen M1 - M16.

\begin{equation}
	\vec{I}=
	\begin{pmatrix}
		- \\
		130.0\,\pm\,1.7 \\
		- \\
		- \\
	  - \\
		- \\
		- \\
		- \\
		122.7\,\pm\,3.1 \\
		141.1\,\pm\,2.2 \\
		122.3\,\pm\,2.0 \\
		-
    -
    -
    -
    -
	\end{pmatrix}
	\sfrac{\text{counts}}{\text{s}}\quad
	\vec{\tilde{I}}=
	\begin{pmatrix}
		130.0\,\pm\,1.7 \\
		130.0\,\pm\,1.7 \\
		130.0\,\pm\,1.7 \\
		130.0\,\pm\,1.7 \\
		130.0\,\pm\,1.7 \\
		130.0\,\pm\,1.7 \\
		122.3\,\pm\,2.0 \\
		141.1\,\pm\,2.2 \\
		122.7\,\pm\,3.1 \\
		141.1\,\pm\,2.2 \\
		122.3\,\pm\,2.0 \\
		122.3\,\pm\,2.0 \\
    141.1\,\pm\,2.2 \\
    122.7\,\pm\,3.1 \\
    141.1\,\pm\,2.2 \\
    122.3\,\pm\,2.0 \\
	\end{pmatrix}
    \sfrac{\text{counts}}{\text{s}}
	\label{eq:int}
\end{equation}

\subsection{Auswertung der Würfel aus einheitlichem Material}

Nach der Anleitung bestehen die beiden Würfel 2 und 3 aus nur einem einzigen Material. Dies ermöglicht, dass die Ergebnisse der verschiedenen, gemessenen Projektionen gemittelt werden können und sich die Geometriematrix zu einem Vektor in den einzelnen Komponenten aufsummieren lässt.
Dadurch folgt für den Würfel 2, der aus Aluminium besteht:

\begin{equation}
	A_2=
	\begin{pmatrix}
		3 \\
		3\cdot\sqrt{2} \\
		2\cdot\sqrt{2} \\
		\sqrt{2}
	\end{pmatrix} \; .
\end{equation}

Für den Würfel 3, der vollständig aus Blei besteht, ergibt sich:

\begin{equation}
	A_3=
	\begin{pmatrix}
		3\cdot\sqrt{2} \\
		2\cdot\sqrt{2} \\
		\sqrt{2}
	\end{pmatrix} \; .
\end{equation}

Diese Matrizen beschreiben die Projektionen 2, 9, 10 und 11, wobei bei dem dritten Würfel die Projektion 9 ausgelassen wurde. Unter Verwendung der
Methode der kleinsten Quadrate lassen sich nun die Absorptionskoeffizienten
der einzelnen Materialien bestimmen. Es ergeben sich die in
Tabelle~\ref{tab:2und3} aufgeführten Ergebnisse.

\begin{table}[htb]
  \centering
  \caption{Aus den verschiedenen Projektionen gemittelte Absorptionskoeffizienten der Würfel 2 und 3.}
  \begin{tabular}{c|
                  S[table-format=1.3(1)]}
    \toprule
    {Objekt} & {Absorptionskoeffizient $\mu$, $\si{\per\centi\meter}$} \\
		\midrule
    Würfel 2 & 0.23(2) \\
    Würfel 3 & 0.93(7) \\
    \bottomrule
  \end{tabular}
  \label{tab:2und3}
\end{table}

\subsection{Auswertung des Würfels unbekannten Aufbaus}

Um den Würfel 5 optimal zu vermessen müssen alle 16 in Abbildung~\ref{fig:wuerfel} dargestellten Projektionen vermessen werden. Denn die Substruktur dieses Würfels ist komplett unbekannt. Es gibt keine Aussagen, sowohl über die verwendeten Materialien, als auch über mögliche Verteilungen dieser innerhalb des Würfels. Somit lassen sich die Absorptionskoeffizienten der neun inneren Würfel unter Verwendung der folgenden Geometriematrix $A$ bestimmen.

\begin{equation}
	A=
	\begin{pmatrix}
    1 & 1 & 1 & 0 & 0 & 0 & 0 & 0 & 0 \\
    0 & 0 & 0 & 1 & 1 & 1 & 0 & 0 & 0 \\
    0 & 0 & 0 & 0 & 0 & 0 & 1 & 1 & 1 \\
    1 & 0 & 0 & 1 & 0 & 0 & 1 & 0 & 0 \\
    0 & 1 & 0 & 0 & 1 & 0 & 0 & 1 & 0 \\
    0 & 0 & 1 & 0 & 0 & 1 & 0 & 0 & 1 \\
    \sqrt{2} & 0 & 0 & 0 & 0 & 0 & 0 & 0 & 0 \\
    0 & \sqrt{2} & 0 & \sqrt{2} & 0 & 0 & 0 & 0 & 0
    0 & 0 & \sqrt{2} & 0 & \sqrt{2} & 0 & \sqrt{2} & 0 & 0 \\
    0 & 0 & 0 & 0 & 0 & \sqrt{2} & 0 & \sqrt{2} & 0 \\
    0 & 0 & \sqrt{2} & 0 & \sqrt{2} & 0 & \sqrt{2} & 0 & 0 \\
    0 & 0 & 0 & 0 & 0 & 0 & \sqrt{2} & 0 & 0 \\
    0 & 0 & 0 & \sqrt{2} & 0 & 0 & 0 & \sqrt{2} & 0 \\
    \sqrt{2} & 0 & 0 & 0 & \sqrt{2} & 0 & 0 & 0 & \sqrt{2} \\
    0 & \sqrt{2} & 0 & 0 & 0 & \sqrt{2} & 0 & 0 & 0 \\
    0 & 0 & \sqrt{2} & 0 & 0 & 0 & 0 & 0 & 0
	\end{pmatrix}
\end{equation}

\begin{equation}
	\vec{I}_5=
	\begin{pmatrix}
		40.2\,\pm\,1.0 \\
		35.4\,\pm\,1.0 \\
		92.2\,\pm\,1.7 \\
		91.4\,\pm\,2.5 \\
		13.2\,\pm\,0.4 \\
    103.6\,\pm\,2.8 \\
		91.8\,\pm\,1.4 \\
		44.8\,\pm\,1.1 \\
		24.5\,\pm\,0.7 \\
		71.9\,\pm\,1.7 \\
		135.7\,\pm\,2.7 \\
    133.6\,\pm\,3.7 \\
		100.7\,\pm\,2.3 \\
		24.6\,\pm\,0.6 \\
		26.7\,\pm\,0.7 \\
		105.5\,\pm\,2.6
	\end{pmatrix}
	\sfrac{\text{counts}}{\text{s}}\quad
	\vec{\tilde{I}}=
	\begin{pmatrix}
    1.2\,\pm\,0.2 \\
		1.3\,\pm\,0.2 \\
		0.3\,\pm\,0.1 \\
		0.4\,\pm\,0.1 \\
		2.3\,\pm\,0.3 \\
    0.2\,\pm\,0.1 \\
		0.3\,\pm\,0.1 \\
		1.0\,\pm\,0.2 \\
		1.7\,\pm\,0.2 \\
		0.5\,\pm\,0.1 \\
		0.1\,\pm\,0.09 \\
    -0.1\,\pm\,0.09 \\
		0.2\,\pm\,0.1 \\
		1.7\,\pm\,0.2 \\
		1.5\,\pm\,0.2 \\
		0.3\,\pm\,0.1
	\end{pmatrix}
	\label{eq:rate5}
\end{equation}

Analog zu den beiden Würfel 2 und 3 liefert die Verwendung der Methode der kleinsten Quadrate für die in
Gleichung~\eqref{eq:rate5} gemessenen Raten und deren normalisierten Logarithmen
die folgenden Absorptionskoeffizienten:

\begin{table}[htb]
  \centering
  \caption{Aus den verschiedenen Projektionen bestimmte Absorptionskoeffizienten der Teilwürfel von Würfel 5.}
  \begin{tabular}{c|
                  S[table-format=1.4(1)]}
    \toprule
    {Teilwürfel} & {Absorptionskoeffizient $\mu$, $\si{\per\centi\meter}$} \\
	\midrule
  Werte = [[ 0.2629  0.8163  0.1509 -0.0456  1.0818  0.1538  0.0117  0.232   0.0059]]
Fehler = [0.057  0.0792 0.0548 0.0626 0.0885 0.0656 0.0488 0.0613 0.0485]
    1 &  0.26(6)\\
    2 &  0.81(8) \\
    3 &  0.15(5) \\
    4 &  -0.05(6) \\
    5 &  1.08(9) \\
    6 &  0.15(7) \\
    7 &  0.01(5) \\
    8 &  0.23(6) \\
    9 &  0.01(5) \\
    \bottomrule
  \end{tabular}
  \label{tab:5}
\end{table}

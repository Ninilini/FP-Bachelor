\section{Diskussion}
In Tabelle \ref{tab:vergl} sind die experimentell bestimmten Landé-Faktoren und die entsprechenden Theoriewerte aufgeführt.
Zudem wurden die Abweichungen bestimmmt. In den statistischen Abweichungen sind lediglich die Fehler durch die Bestimmung der
Magnetfeldstärke berücksichtigt worden, weitere Fehler konnten nicht bestimmt werden. Auffällig ist, dass die Abweichungen der
roten $\sigma$-Linie und der blauen $\pi$-Linie gering sind, während die Abweichung der blauen $\sigma$-Linie mit $\SI{20.6}{\percent}$
vergleichsweise hoch ausfällt.

Gründe für Abweichungen können sowohl systematische Fehler im Versuchsaufbau als auch Ungenauigkeiten in
der Auswertung der Bilder sein. Problematisch könnte zum einen die Eichung des Magnetfelds sein, da die Hall-Sonde nicht fest befestigt wurde,
sondern nur per Hand ins Magnetfeld gehalten wurde. Da diese sehr empfindlich bezüglich Winkeländerungen gegenüber des Magnetfelds ist, kann die
Messung ungenau sein. Bei der Auswertung der Fotos wurde das Intensitätsmaximum lediglich abgeschätzt und nicht ausgemessen, sodass dabei große
Ungenauigkeiten entstanden sein können. Insbesondere bei der blauen $\sigma$-Linie im Magnetfeld war die Abschätzung des Maximums schwierig, was
die Hauptursache für die große Abweichung bei dieser Linie sein kann. 

\begin{table}
  \centering
  \caption{Vergleich der experimentell bestimmten Werte mit den Theoriewerten.}
  \label{tab:vergl}
  \begin{tabular}{c S[table-format=1.3] @{${}\pm{}$} S[table-format=1.3] S[table-format=1.2] S[table-format=2.1]}
    \toprule
    Polarisation & \multicolumn{2}{c}{$g_\text{exp}$} & {$g_\text{theo}$} & {$\upD g \:/\: \si{\percent}$} \\
    \midrule
    $\sigma_\text{rot}$  & 0.981 & 0.009         & 1    &  1.9 \\
    $\pi_\text{rot}$     & \multicolumn{2}{c}{0} & 0    &  {-} \\
    $\sigma_\text{blau}$ & 1.39  & 0.02          & 1.75 & 20.6 \\
    $\pi_\text{blau}$    & 0.525 & 0.004         & 0.5  &  5.0 \\
    \bottomrule
  \end{tabular}
\end{table}

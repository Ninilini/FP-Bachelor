\section{Auswertung}
\subsection{Fehlerrechnung}
\label{sec:auswertung}
Die Mittelwerte der für die Auswertung mehrfach gemessenen Größen sind
gemäß der Gleichung
\begin{equation}
    \bar{x}=\frac{1}{n}\sum_{i=1}^n x_i
    \label{eq:mittelwert}
\end{equation}
bestimmt. Die Standardabweichung des Mittelwertes ergibt sich dabei zu
\begin{equation}
    \mathup{\Delta}\bar{x}=\sqrt{\frac{1}{n(n-1)}\sum_{i=1}^n\left(x_i-\bar{x}\right)^2}.
    \label{eq:standardabweichung}
\end{equation}
Existiert eine Größe, die sich durch die Berechnung aus zwei Fehler behafteten Größen ergibt,
so berechnet sich der Gesamtfehler mit Hilfe der Gaußschen Fehlerfortpflanzung zu
\begin{equation}
    \mathup{\Delta}f(x_1,x_2,...,x_n)=\sqrt{\left(\frac{\partial f}{\partial x_1}\mathup{\Delta}x_1\right)^2+\left(\frac{\partial f}{\partial x_2}\mathup{\Delta}x_2\right)^2+ \dotsb +\left(\frac{\partial f}{\partial x_n}\mathup{\Delta}x_n\right)^2}.
    \label{eq:fehlerfortpflanzung}
\end{equation}
Die in der Auswertung berechneten Größen sind alle  auf die erste signifikante
Stelle des Fehlers gerundet. Setzt sich eine Größe über mehrere Schritte aus
anderen Größen zusammen, so wird erst am Ende gerundet, um Fehler zu vermeiden.
% Zur Auswertung wird die Programmiersprache \texttt{python (Version 3.4.1)} mit
% den Bibliothekserweiterungen \texttt{numpy}~\cite{numpy},
% \texttt{scipy}~\cite{scipy} und \texttt{matplotlib}~\cite{matplotlib} zur
% Erstellung der Grafiken und linearen Regressionen verwendet.

\subsection{Bestimmung der Molwärme von Kupfer}
In der folgenden Auswertung werden die Fehler von Größen immer durch
ein~$\upDelta$ kenntlich gemacht. Für die benötigten Zeit-, Temperatur- und
Energiedifferenzen wird ein~$\delta$ verwendet.\\
\newline
Die molare Wärmekapazität~$C$ eines Stoffes berechnet sich im Experiment aus
dem Differenzenquotienten
\begin{equation}
  C=\frac{\delta Q}{m \cdot \delta T},
\end{equation}
Durch die Heizwicklung wird der Kupferprobe kontinuierlich eine elektrische Energie~$\delta E$
zugeführt, sodass eine Wärmemenge
\begin{equation}
  \delta Q = UI\delta t
\end{equation}
aufgenommen wird. Dabei entsprechen $U$ der Spannung und $I$ dem Strom der Heizwicklung.
Das gemessene Zeitintervall beträgt $\delta t=\SI{300(1)}{\second}$ beziehungsweise $\delta t=\SI{310(1)}{\second}$.
Der angegebene Fehler resultiert aus der Ablesegenauigkeit des verwendeten Zeitmessers. Die benötigte
Temperaturdifferenz~$\delta T$ wird aus den gemessenen Widerständen~$R_1 \text{und}\, R_2$ bestimmmt.
Die zugehörige Umrechnungsformel lautet
\begin{gather}
  T=\SI{0.00134}{\kelvin\per\ohm\squared}R^2+\SI{2.296}{\kelvin\per\ohm}R+\SI{30.13}{\kelvin} \\
  \shortintertext{mit dem Fehler}
  \upDelta T=\left(\SI{0.00268}{\kelvin\per\ohm}+\SI{2.296}{\kelvin}\right)\SI{}{\per\ohm}\upDelta R
\end{gather}
und ist~\cite{anleitung} entnommen. Es soll die Wärmekapazität bei konstantem
Volumen~$C_{\mathrm{V}}$ berechnet werden. Da temperaturabhängige Messungen bei
konstantem Volumen an einem Festkörper jedoch schwer durchzuführen sind, wird
die Messung stattdessen bei konstantem Druck durchgeführt.
Die auf die Stoffmenge~$n=\sfrac{m}{M}$ bezogene molare Wärmekapazität bei
konstantem Druck~$C_{\mathrm{P}}$ ergibt sich zu
\begin{gather}
  C_{\mathrm{P}}=\frac{M}{m}\frac{UI\delta t}{\delta T}\\
  \intertext{mit dem Fehler}
  \upDelta C_{\mathrm{P}}=\frac{M}{m}\sqrt{\left(\frac{I\delta t}{\delta T}\upDelta U\right)^2+\left(\frac{U\delta t}{\delta T}\upDelta I\right)^2+\left(\frac{UI}{\delta T}\upDelta(\delta t)\right)^2+\left(\frac{UI\delta t}{(\delta T)^2}\upDelta(\delta T)\right)^2}.
\end{gather}
Hierbei ist~$M=\SI{63.546}{\gram\per\mol}$~\cite{mathematica} die molare Masse
von Kupfer und~$m=\SI{0.342}{\kilo\gram}$~\cite{anleitung} die Masse der Probe. Die
Umrechnung von~$C_{\mathrm{P}}$ in~$C_{\mathrm{V}}$ erfolgt mit Hilfe der
Gleichung
\begin{gather}
  C_{\mathrm{V}}=C_{\mathrm{P}}-9\alpha^2\kappa V_0T\label{eq:Cp2Cv} \\
  \intertext{und hat den Fehler}
  \upDelta C_{\mathrm{V}}=\sqrt{(\upDelta C_{\mathrm{P}})^2+(18\alpha\kappa V_0T\upDelta\alpha)^2+(9\alpha^2\kappa V_0\upDelta T)^2}.
\end{gather}
Das Kompressionsmodul~$\kappa$ von Kupfer beträgt nach~\cite{mathematica}
\begin{equation}
  \kappa=\SI{140}{\giga\newton\per\metre\squared}.
\end{equation}
Das molare Volumen~$V_0$ der verwendeten Probe berechnet sich mit Hilfe der
Avogadro-Konstanten~$N_{\mathrm{A}}=\SI{6.022e23}{\per\mol}$ aus der
Dichte~$\rho=\SI{8.960}{\gram\per\centi\metre\cubed}$~\cite{mathematica} und der
molaren Masse~$M$ zu
\begin{equation}
  V_0=\frac{M}{\rho}=\SI{7.092e-6}{\metre\cubed\per\mol}.
\end{equation}
$\alpha$ ist der temperaturabhängige lineare Ausdehnungskoeffizient, für den in
Tabelle~\ref{tab:ausdehnungskoeffizient} einige Werte bei verschiedenen
Temperaturen angegeben sind.

\begin{table}[htb]
    \centering
    \tiny
    \caption{Linearer Ausdehnungskoeffizient~$\alpha$ von Kupfer in Abhängigkeit
    der Temperatur~$T$ \cite{anleitung}.}
    \begin{tabular}{S[table-format = 3.0]
                    S[table-format = 2.2] |
                    S[table-format = 3.0]
                    S[table-format = 2.2] }
        \toprule
        {$T$ in \si{\kelvin}} & {$\alpha\cdot 10^{-6}$ in \si{\per\kelvin}} & {$T$ in \si{\kelvin}} & {$\alpha\cdot 10^{-6}$ in \si{\per\kelvin}} \\
        \midrule
         70 &  7.00 & 190 & 14.75 \\
         80 &  8.50 & 200 & 14.95 \\
         90 &  9.75 & 210 & 15.20 \\
        100 & 10.70 & 220 & 15.40 \\
        110 & 11.50 & 230 & 15.60 \\
        120 & 12.10 & 240 & 15.75 \\
        130 & 12.65 & 250 & 15.90 \\
        140 & 13.15 & 260 & 16.10 \\
        150 & 13.60 & 270 & 16.25 \\
        160 & 13.90 & 280 & 16.35 \\
        170 & 14.25 & 290 & 16.50 \\
        180 & 14.50 & 300 & 16.65 \\
        \bottomrule
    \end{tabular}
    \label{tab:ausdehnungskoeffizient}
\end{table}

Da $\alpha$ von Festkörpern eine~$T^{-1}$-Abhängigkeit besitzt, lässt sich ein linearer Fit der Form
\begin{gather}
  \alpha(T)=m\frac{1}{T}+b \\
  \shortintertext{mit dem Fehler}
  \upDelta\alpha(T,\upDelta T)=\sqrt{\left(\frac{1}{T}\upDelta m\right)^2+\left(\frac{m}{T^2}\upDelta T\right)^2+(\upDelta b)^2}
\end{gather}
durchführen. Dies dient zur Bestimmung des Ausdehnungskoeffizienten zwischen den gegebenen Stützstellen.
Der Fit ist in Abbildung~\ref{fig:plot_alpha} dargestellt\\
\newline
\begin{figure}[htb]
    \centering
    \includegraphics[width=0.8\textwidth]{plots/plot_alpha.pdf}
    \caption{Linearer Ausdehnungskoeffizient~$\alpha$ von Kupfer aufgetragen
    gegen die inverse Temperatur.}
    \label{fig:plot_alpha}
\end{figure}
und liefert für die Parameter~$m$ und~$b$ die Werte
\begin{gather}
  m = \SI{-873(4)e-6}{} \\
  \shortintertext{und}
  b = \SI{19.41(3)e-6}{\per\kelvin}.
\end{gather}
Tabelle~\ref{tab:Molwaerme} zeigt zusammenfassend alle Messdaten
sowie die zugehörigen berechneten Größen Temperatur und die molare Wärmekapazität~$C_{\mathrm{P}}$.
In Abbildung~\ref{fig:plot_Cv} sind
die berechneten Werte für~$C_{\mathrm{V}}$ samt Fehlerbalken gegen die Temperatur aufgetragen dargestellt.\\

\begin{figure}[htb]
    \centering
    \includegraphics[width=\textwidth]{plots/plot_Cv.pdf}
    \caption{Berechnete Werte für~$C_{\mathrm{P}}$ und~$C_{\mathrm{V}}$
    aufgetragen gegen die Temperatur. Im Bereich zwischen~\SI{100}{\kelvin}
    und~\SI{170}{\kelvin} ist der Verlauf annähernd linear.}
    \label{fig:plot_Cv}
\end{figure}

\subsection{Bestimmung der Debye-Temperatur}
Ziel des Versuchs ist es zuletzt die Debye-Temperatur~$\theta_{\mathrm{D}}$ der Kupferprobe zu
bestimmen. Dafür werden aus Tabelle~1 in~\cite{anleitung} die Quotienten aus der
Debye-Temperatur und der gemessenen Temperatur~$T$ abgelesen und durch
Multiplikation mit~$T$ wird anschließend die Debye-Temperatur bestimmt. Es werden nur
Werte~$C_{\mathrm{V}}$ betrachtet, bei denen für die Temperatur gilt~$T<\SI{170}{\kelvin}$.
Die errechneten Werte für~$\theta_{\mathrm{D}}$ sind in
Tabelle~\ref{tab:debyetemperatur} zusammengefasst. Als Mittelwert ergibt sich
mit Formel~\eqref{eq:mittelwert} und Formel~\eqref{eq:fehlerfortpflanzung}
\begin{equation}
  \theta_{\mathrm{D},1}=\SI{366.45(23)}{\kelvin}.
\end{equation}

\begin{table}[H]
    \centering
    \caption{Gemessene und berechnete physikalische Größen zur Bestimmung der
    Debye-Temperatur~$\theta_{\mathrm{D}}$ einer Kupferprobe.}
    \begin{tabular}{S[table-format=3.1(1)]
                    S[table-format=2.1(1)]
                    S
                    S[table-format=3.1(1)]}
        \toprule
        {$T_2$ in \si{\kelvin}} & {$C_{\mathrm{V}}$ in \si{\joule\per\mol\per\kelvin}} & {$\frac{\theta_{\mathrm{D}}}{T}$} & {$\theta_{\mathrm{D}}$ in \si{\per\kelvin}} \\
        \midrule
         91.4(2) & 12.8(5) & 3.9 & 356.6(9) \\
         98.3(2) & 16.1(8) & 3.1 & 304.8(7) \\
        106.6(2) & 16.7(8) & 3.0 & 319.9(7) \\
        112.6(2) & 20.0(10) & 2.2 & 247.7(5) \\
        119.6(2) & 19.9(10) & 2.2 & 263.0(5) \\
        129.2(2) & 17.6(8) & 2.7 & 348.7(7) \\
        137.4(2) & 26.2(10) & 0.0 & 0.0(0) \\
        147.3(2) & 14.1(5) & 3.6 & 530.3(9) \\
        158.3(2) & 12.8(5) & 3.9 & 617.2(10) \\
        169.0(2) & 12.7(5) & 4.0 & 676.1(10) \\
        \bottomrule
    \end{tabular}
    \label{tab:debyetemperatur}
\end{table}

Ein alternativer Ansatz ermöglicht die Berechnung der Debye-Temperatur aus
Forderung~\eqref{eq:konv}. Es ergibt sich
\begin{equation}
  \theta_{\mathrm{D},2}=\frac{\hbar}{k_{\mathrm{B}}}\sqrt[3]{\frac{18\pi^2N_A\rho}{M}\left(v_{\mathrm{long}}^{-3}+2v_{\mathrm{trans}}^{-3}\right)^{-1}}=\SI{332,5}{\kelvin},
\end{equation}
wobei~$v_{\mathrm{long}}=\SI{4.7}{\kilo\metre\per\second}$ und $v_{\mathrm{trans}}=\SI{2.26}{\kilo\metre\per\second}$ verwendet wurde.
Für die Debye-Frequenzen ergeben sich die Werte
\begin{align}
  \omega_{\mathrm{D},1}&=\frac{k_{\mathrm{B}}}{\hbar}\theta_{\mathrm{D},1}=\SI{47.98(3)}{\tera\hertz} \\
  \shortintertext{und}
  \omega_{\mathrm{D},2}&=\frac{k_{\mathrm{B}}}{\hbar}\theta_{\mathrm{D},2}=\SI{43.53}{\tera\hertz}.
\end{align}

\begin{sidewaystable}[htb]
  \centering
  \caption{Gemessene und berechnete physikalische Größen zur Bestimmung der
  molaren Wärmekapazität einer Kupferprobe.}
  \begin{tabular}{S[table-format=2.2,separate-uncertainty,table-figures-uncertainty=1]
                  S[table-format=2.2,separate-uncertainty,table-figures-uncertainty=1]
                  S[table-format=2.2,separate-uncertainty,table-figures-uncertainty=1]
                  S[table-format=2.2,separate-uncertainty,table-figures-uncertainty=1]
                  S[table-format=2.2,separate-uncertainty,table-figures-uncertainty=1]
                  S[table-format=2.2,separate-uncertainty,table-figures-uncertainty=1]
                  S[table-format=2.2,separate-uncertainty,table-figures-uncertainty=1]
                  S[table-format=2.2,separate-uncertainty,table-figures-uncertainty=1]}
      \toprule
      {$\delta t$ in \si{\second}} & {$U$ in \si{\volt}} & {$I$ in \si{\milli\ampere}} & {$R1$ in \si{\ohm}} & {$R2$ in \si{\ohm}} & {$T1$ in \si{\kelvin}} & {$T2$ in \si{\kelvin}} & {$C_{\mathrm{P}}$ in \si{\joule\per\mol\per\kelvin}} \\
      \midrule
      0(0)    &   0(0) &  0(0)    &  0(0)   &  22.7(1)  & 0(0)       & 82.94(24)&	0(0)\\
      300(1)& 12.80(1) & 152.4(1) & 22.7(1) &  26.3(1)  & 82.94(23)  & 91.44(24)&	12.79(50)\\
      300(1)& 12.81(1) & 154.7(1) & 26.3(1) &  29.2(1)  & 91.44(24)  & 98.32(24)&	16.07(79)\\
      300(1)& 12.85(1) & 166.3(1) & 29.7(1) &  32.7(1)  & 99.50(24)  & 106.64(24)&	16.69(79)\\
      300(1)& 12.86(1) & 166.6(1) & 32.7(1) &  35.2(1)  & 106.64(24) & 112.61(24)&	20.01(113)\\
      300(1)& 12.91(1) & 178.9(1) & 35.4(1) &  38.1(1)  & 113.09(24) & 119.55(24)&	19.91(105)\\
      310(1)& 12.94(1) & 187.0(1) & 38.8(1) &  42.1(1)  & 121.23(24) & 129.17(24)&	17.57(76)\\
      310(1)& 19.95(1) & 187.4(1) & 42.1(1) &  45.5(1)  & 129.17(24) & 137.37(24)&	26.24(110)\\
      310(1)& 12.96(1) & 187.6(1) & 45.5(1) &  49.6(1)  & 137.37(24) & 147.31(24)&	14.09(49)\\
      310(1)& 12.98(1) & 187.9(1) & 49.6(1) &  54.1(1)  & 147.31(24) & 158.27(24)&	12.82(41)\\
      300(1)& 12.99(1) & 188.1(1) & 54.1(1) &  58.5(1)  & 158.27(24) & 169.03(25)&	12.65(41)\\
      300(1)& 13.01(1) & 188.2(1) & 58.5(1) &  62.5(1)  & 169.03(25) & 178.86(25)&	13.88(49)\\
      300(1)& 13.03(1) & 188.3(1) & 62.5(1) &  67.2(1)  & 178.86(25) & 190.47(25)&	11.78(36)\\
      300(1)& 13.04(1) & 188.4(1) & 67.2(1) &  69.7(1)  & 190.47(25) & 196.67(25)&	22.09(125)\\
      300(1)& 13.06(1) & 188.4(1) & 69.7(1) &  73.1(1)  & 196.67(25) & 205.13(25)&	16.22(68)\\
      300(1)& 13.08(1) & 188.5(1) & 73.1(1) &  76.6(1)  & 205.13(25) & 213.87(25)&	15.73(64)\\
      300(1)& 13.09(1) & 188.6(1) & 76.6(1) &  80.3(1)  & 213.87(25) & 223.14(25)&	14.84(57)\\
      300(1)& 13.11(1) & 188.7(1) & 80.3(1) &  83.8(1)  & 223.14(25) & 231.94(25)&	15.66(64)\\
      300(1)& 13.12(1) & 188.8(1) & 83.8(1) &  86.9(1)  & 231.94(25) & 239.77(25)&	17.64(81)\\
      310(1)& 13.14(1) & 188.8(1) & 86.9(1) &  90.0(1)  & 238.77(25) & 247.62(25)&	18.20(83)\\
      310(1)& 13.16(1) & 188.9(1) & 90.0(1) &  93.4(1)  & 247.62(25) & 256.27(25)&	16.57(69)\\
      310(1)& 13.17(1) & 188.9(1) & 93.4(1) &  97.0(1)  & 256.27(25) & 265.45(26)&	15.60(62)\\
      310(1)& 13.19(1) & 189.0(1) & 97.0(1) & 100.5(1)  & 265.45(26) & 274.41(26)&	16.02(65)\\
      310(1)& 13.20(1) & 189.0(1) &100.5(1) & 103.7(1)  & 274.41(26) & 282.64(26)&	17.48(77)\\
      310(1)& 13.21(1) & 189.9(1) &103.7(1) & 106.5(1)  & 282.64(26) & 289.85(26)&	20.02(101)\\
      310(1)& 13.22(1) & 189.1(1) &106.5(1) & 109.1(1)  & 289.85(26) & 296.57(26)&	21.43(117)\\
      300(1)& 13.23(1) & 189.1(1) &109.1(1) & 111.8(1)  & 296.57(26) & 303.57(26)&	19.93(105)\\
      \bottomrule
  \end{tabular}
  \label{tab:Molwaerme}
\end{sidewaystable}

\section{Diskussion}
\label{sec:diskussion}
Der Literaturwert für die Debye-Temperatur
beträgt~$\theta_{\mathrm{D,lit}}=\SI{343}{\kelvin}$~\cite{kittel}.
Es zeigt sich, dass dieser nur um ungefähr~\SI{3.1}{\percent}
von~$\theta_{\mathrm{D,2}}$ abweicht, welcher mit Hilfe theoretischer
Überlegungen aus Forderung~\eqref{eq:konv} bestimmt wurde.
Die Abweichung des im Versuch gemessenen
Wertes~$\theta_{\mathrm{D,1}}$ vom Literaturwert beträgt
rund~\SI{6.8}{\percent}. Mögliche Fehlerquellen dabei sind, dass bei Beginn der Messung
der absolute Nullpunkt nur annährend erreicht werden kann, genauso wie der zum Kühlen
verwendete Stickstoff sehr schnell verdampft. Dies führt zum einen dazu, dass mögliche
Wärmeverluste nicht mehr so stark unterdrückt werden und zum anderen lösen sich mögliche
Moleküle von den außen Wänden ab, die vorher festgefroren sind. Der zweite Effekt soll
dadurch verringert werden, dass die Vakuumpumpe während des Gesamtenversuchs läuft. Er
würde sonst zu Wärmeverluste durch Wärmetransport führen.
Die größte Fehlerquelle stellt jedoch der Temperaturgradient zwischen Rezipient und Probe
dar. Dieser ist zwischenzeitlich so groß geworden, dass die Messung gestört wurde. Dies ist
besonders deutlich in Tabelle \ref{tab:debyetemperatur} zusehen. Bei dem siebten Eintrag kommt
es zu so großen Schwankungen, dass der Messwert nicht mehr im Rahmen der Tabelle zur Bestimmmung
der Debyetemperatur liegt. Dies lässt sich wieder auf Wärmestrahlung zurückführen und entsteht, da
die Regulierung für die Heizwicklung extrem träge reagiert.
Zudem kommt es bei der Auslesung der Widerstands-, Strom- und Spannungswerte zu Ablesefehlern,
diese könnten durch ein automatisiertes Verfahren minimiert werden.

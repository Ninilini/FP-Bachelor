\section{Zielsetzung}
In diesem Versuch soll die Molwärme von Kupfer bestimmt werden. Aus dieser wird
dann die Debye-Temperatur bestimmt und mit dem Literaturwert verglichen.

\section{Theorie}
Zunächst werden drei Theorien zur Molwärme von Festkörpern erläutert und verglichen.
\subsection{Die klassische Theorie}
Nach der klassischen Physik können die Gitteratome eines Festkörpers in drei orthogonale
Richtungen um ihre Ruhelage schwingen. Da es sich um harmonische Schwingungen handelt, sind
dabei die mittlere kinetische Energie und die mittlere potentielle Energie gleich. Nach dem
Äquipartitionstheorem erhält jedes Atom pro Freiheitsgrad die mittlere Wärmeenergie $\frac{1}{2}k_\text{B}T$,
sodass sich pro Mol die Energie
\begin{equation}
  U = 3RT
\end{equation}
mit der allgemeinen Gaskonstante $R$ ergibt. Daraus folgt die spezifische Molwärme bei konstantem Volumen
\begin{equation}
  C_\text{V} = \left(\frac{\partial U}{\partial T}\right)_\text{V} = 3R \,.
\end{equation}
Diese ist materialunabhängig und ebenfalls unabhängig von der Temperatur, obwohl dies im Widerspruch zu experimentellen
Ergebnissen steht. Nur bei hohen Temperaturen ($T \geq \theta_\text{D}$) nähert sich die spezifische Molwärme gemäß
dem Dulong-Petitschen Gesetz dem Wert $3R$ an.

\subsection{Das Einstein-Modell}
Das Einstein-Modell berücksichtigt im Gegensatz zum klassischen Modell die quantenmechanische Energiequantelung. Allerdings
wird angenommen, dass alle Atome mit der gleichen Frequenz schwingen und Energiebeträge abgeben und aufnehmen können, die einem
Vielfachen von $\hbar\omega$ entsprechen.

Mit der Boltzmann-Verteilung
\begin{equation*}
  W(n) = \exp\left( -\frac{n\hbar\omega}{k_\text{B}T} \right)
\end{equation*}
für die Wahrscheinlichkeit, dass ein Oszillator die Energie $n\hbar\omega$ hat, ergibt sich die mittlere Energie pro Atom
\begin{equation}
  \langle u \rangle_\text{Einstein} = \frac{\hbar\omega}{\exp(\hbar\omega/k_\text{B}T) - 1} < k_\text{B}T \,.
\end{equation}
Daraus folgt für die Molwärme:
\begin{equation}
  C_\text{V, Einstein} = 3R\frac{\hbar^2\omega^2}{k_\text{B}^2T^2}
                          \frac{\exp(\hbar\omega/k_\text{B}T)}{\left[\exp(\hbar\omega/k_\text{B}T)\right]^2} \,.
\end{equation}
Diese erfüllt ebenfalls das Dulong-Petitsche Gesetz
\begin{equation*}
  \lim_{T\to\infty} C_\text{V, Einstein} = 3R\,.
\end{equation*}
Im Gegensatz zum klassischen Ansatz zeigt das Einstein-Modell die beobachtete Temperaturabhängigkeit. Allerdings weicht das
Einstein-Modell bei tiefen Temperaturen dennoch von den experimentellen Ergebnissen ab, da entgegen der Annahme
nicht nur Schwingungen einer Frequenz auftreten.

\subsection{Das Debye-Modell}
Das Debye-Modell berücksichtigt die spektrale Verteilung $Z(\omega)$ der Eigenschwingungen. Diese Verteilung kann bei einigen
Kristallen allerdings sehr kompliziert sein und deshalb wird als Näherung angenommen, dass die Phasengeschwindigkeit
unabhängig von der Frequenz und der Ausbreitungsrichtung der Welle im Kristall ist, sodass sich $Z(\omega)$ einfach berechnen lässt.

%%%%%%%%%%%%%%%%%%%%%%%%%%%%%%
% Erklärung Debye-Frequenz?!
%%%%%%%%%%%%%%%%%%%%%%%%%%%%%%

Für die spezifische Molwärme ergibt sich schließlich
\begin{equation}
  C_\text{V, Debye} = 9R \left(\frac{T}{\theta_\text{D}}\right)^3 \int_0^{\theta_\text{D}/T} \frac{x^4\text{e}^x}{(\text{e}^x-1)^2}\,\text{d}x
\end{equation}
mit
\begin{equation*}
  x = \frac{\hbar\omega}{k_\text{B}T} \qquad \text{und} \qquad \frac{\theta_\text{D}}{T} = \frac{\hbar\omega_\text{D}}{k_\text{B}T}\,.
\end{equation*}

$\theta_\text{D}$ ist dabei die materialspezifische Debye-Temperatur. Auch im Debye-Modell gilt
\begin{equation*}
  \lim_{T\to\infty} C_\text{V, Debye} = 3R \,.
\end{equation*}

%%%%%%%%%%%%%%%%%%%%%%%%%%%%
% Tieftemperaturverhalten
%%%%%%%%%%%%%%%%%%%%%%%%%%%%
